% !TEX TS-program = xelatex
% !TEX encoding = UTF-8

% This is a simple template for a XeLaTeX document using the "article" class,
% with the fontspec package to easily select fonts.

\documentclass[11pt]{article} % use larger type; default would be 10pt

\usepackage{zhfontcfg}
\setmainfont{Myriad Pro}
\setCJKmainfont{Hiragino Sans GB}

\usepackage{geometry} % See geometry.pdf to learn the layout options. There are lots.
\geometry{a4paper} % or letterpaper (US) or a5paper or....
%\usepackage[parfill]{parskip} % Activate to begin paragraphs with an empty line rather than an indent

\usepackage{graphicx} % support the \includegraphics command and options

\title{Light C Internal}
\author{ TianXiao Xia}
\date{} % Activate to display a given date or no date (if empty),
         % otherwise the current date is printed 

\begin{document}
\maketitle
LightC is a ANSI C compiler.  implement with modern compiler-kits: yacc lex llvm. 

为了实践编译原理,解决实际实现中会遇到的问题;希望能做到结构清晰,即使会牺牲些性能

%%%%
\section{Architecture}


取舍:
\begin{itemize}
\item 构造出AST后再生成中间代码,而不是一边parse一边生成
\item 使用llvmtype时,包裹一个wrap类,方便以后替换掉llvm的东西
\item 构造AST时完成声明和符号表的处理,AST中的decl节点含有中间代码的类型信息
\end{itemize}

\begin{itemize}
\item 为单条声明语句作准备
\end{itemize}

\begin{itemize}
\item 引入轻量级的string代替std::string,类似llvm::StringRef

\end{itemize}
%%%%
\section{Grammar of C}
\begin{itemize}
\item declaration 一条声明语句,可以是变量声明、函数声明、for循环声明或其他复杂声明
\item primary 包括变量、整数、字符串等
\item declarator 声明中得变量,被声明对象
\end{itemize}

The Mininal Subset of C

func_define,int type,

%%%%
\section{Type subsystem}
getInt32Ty的实现,一个impl类
%%%%
\section{Memory}

%%%%
\section{AST internal}
three kind of AST: Type Decl Stmt

although could just use llvm's Type system,but attemp approach another way
\subsection{SymbolTable}
符号表由链表链接
\begin{itemize}
\item 在\{\}的进入和退出时执行push和pop,由词法解析器完成,symboltable放在compound\_statement里
\item function\_definition 结束时pop
\item function proto 结束时 push
\item declaration in for\-loop 时处理和函数处理相同
\item declaration in block\_item 直接加入当前符号表
\item declaration 同上(此处是函数外的全局声明)
\end{itemize}

\subsection{Implement}
词法分析器在'\{、 \}'时,分别执行push和pop操作

借助于一个标志变量,proto在第一个参数开始的时候push,直到整个函数结束的时候再Pop,对于abstract声明或纯声明未测试

全局有一个context存放全局变量


%globale[ [declaration] proto[ [declaration] func_def[declaration] ] ]
%%%%
\section{中间代码}
\subseciton{函数参数处理}
llvm::IR的函数参数,value是用Functin::Iterator = arg\_begin()得到的
\subsection{How to represent stack variables}
declaration分为global var和function var,有不同的生成代码

\begin{itemize}
\item mutable local variables, same as clang and llvm-gcc
\end{itemize}

\section{Error Recovery}
ErrorV的设计很巧妙

\section{Testing}
生成的IR代码与clang的IR代码对比,优化选项全部关闭


根据llvm 的cpp后端,生成IR产生代码,进行调试
\section{ChangLog}
开发计划 

先支持函数,基本运算,变量定义;建立命名空间,类型系统,代码生成框架,内存处理;然后再逐步扩展语言

load store完成,可以合并到master分支了,准备开始handwrite部分

参考资料:

crafting a compiler with c



\section{Index}
\begin{itemize}
\item SSA Single Static Assignment subset of CPS
\item CPS Continuation Pass Style for functional language
\end{itemize}
\end{document}
